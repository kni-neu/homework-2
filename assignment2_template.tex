%%%%%%%%%%%%%%%%%%%%%%%%%%%%%%%%%%%%%%%%%
% Short Sectioned Assignment
% LaTeX Template
% Version 1.0 (5/5/12)
%
% This template has been downloaded from:
% http://www.LaTeXTemplates.com
%
% Original author:
% Frits Wenneker (http://www.howtotex.com)
%
% License:
% CC BY-NC-SA 3.0 (http://creativecommons.org/licenses/by-nc-sa/3.0/)
%
%%%%%%%%%%%%%%%%%%%%%%%%%%%%%%%%%%%%%%%%%

%----------------------------------------------------------------------------------------
%	PACKAGES AND OTHER DOCUMENT CONFIGURATIONS
%----------------------------------------------------------------------------------------

\documentclass[paper=a4, fontsize=11pt]{scrartcl} % A4 paper and 11pt font size
\usepackage[top=1in, bottom=1.5in, left=1in, right=1in]{geometry}
\usepackage{fancyhdr} % Required for custom headers
\usepackage{lastpage} % Required to determine the last page for the footer
\usepackage{extramarks} % Required for headers and footers
\usepackage[usenames,dvipsnames]{color} % Required for custom colors
\usepackage{graphicx} % Required to insert images
\usepackage{listings} % Required for insertion of code
\usepackage{courier} % Required for the courier font
\usepackage{amsmath}
\usepackage[super]{nth}
\usepackage{booktabs}
\usepackage[usenames,dvipsnames]{xcolor}
\usepackage{tcolorbox}
\usepackage{tabularx}
\usepackage{array}
\usepackage{colortbl}

%\usepackage[T1]{fontenc} % Use 8-bit encoding that has 256 glyphs
%\usepackage{fourier} % Use the Adobe Utopia font for the document - comment this line to return to the LaTeX default
\usepackage[english]{babel} % English language/hyphenation
\usepackage{amsmath,amsfonts,amsthm} % Math packages
\usepackage{graphicx}

\usepackage{hyperref}
\hypersetup{
  colorlinks   = true, %Colours links instead of ugly boxes
  urlcolor     = blue, %Colour for external hyperlinks
  linkcolor    = blue, %Colour of internal links
  citecolor   = red %Colour of citations
}

\usepackage{fancyhdr} % Custom headers and footers
\pagestyle{fancyplain} % Makes all pages in the document conform to the custom headers and footers
\fancyhead{} % No page header - if you want one, create it in the same way as the footers below
\fancyfoot[L]{} % Empty left footer
\fancyfoot[C]{} % Empty center footer
\fancyfoot[R]{\thepage} % Page numbering for right footer
\renewcommand{\headrulewidth}{0pt} % Remove header underlines
\renewcommand{\footrulewidth}{0pt} % Remove footer underlines
\setlength{\headheight}{13.6pt} % Customize the height of the header
\newcommand{\ts}{\textsuperscript}

\numberwithin{equation}{section} % Number equations within sections (i.e. 1.1, 1.2, 2.1, 2.2 instead of 1, 2, 3, 4)
\numberwithin{figure}{section} % Number figures within sections (i.e. 1.1, 1.2, 2.1, 2.2 instead of 1, 2, 3, 4)
\numberwithin{table}{section} % Number tables within sections (i.e. 1.1, 1.2, 2.1, 2.2 instead of 1, 2, 3, 4)

\setlength\parindent{0pt} % Removes all indentation from paragraphs - comment this line for an assignment with lots of text

% Default fixed font does not support bold face
\DeclareFixedFont{\ttb}{T1}{txtt}{bx}{n}{8} % for bold
\DeclareFixedFont{\ttm}{T1}{txtt}{m}{n}{8}  % for normal

%----------------------------------------------------------------------------------------
%	CODE BLOCKS
%----------------------------------------------------------------------------------------

\usepackage{adjustbox}
\usepackage{listings}
\usepackage{color}

\definecolor{dkgreen}{rgb}{0,0.6,0}
\definecolor{gray}{rgb}{0.5,0.5,0.5}
\definecolor{mauve}{rgb}{0.58,0,0.82}

\lstdefinelanguage{Dockerfile}
{
  morekeywords={FROM, RUN, CMD, LABEL, MAINTAINER, EXPOSE, ENV, ADD, COPY,
    ENTRYPOINT, VOLUME, USER, WORKDIR, ARG, ONBUILD, STOPSIGNAL, HEALTHCHECK,
    SHELL},
  morecomment=[l]{\#},
  morestring=[b]"
}

\lstset{
    columns=flexible,
    aboveskip=5mm,
    belowskip=5mm,
    keepspaces=true,
    showstringspaces=false,
    basicstyle=\ttfamily,
    commentstyle=\color{gray},
    keywordstyle=\color{purple},
    stringstyle=\color{green}
}



%----------------------------------------------------------------------------------------
%	TITLE SECTION
%----------------------------------------------------------------------------------------

\usepackage{eso-pic}
% \usepackage[demo]{graphicx}
\newcommand\AtPageUpperRight[1]{\AtPageUpperLeft{%
   \makebox[\paperwidth][r]{#1}}}

\newcommand{\horrule}[1]{\rule{\linewidth}{#1}} % Create horizontal rule command with 1 argument of height

\title{	
\normalfont \normalsize
\textsc{Northeastern University,  Khoury College of Computer Science} \\ [25pt] % Your university, school and/or department name(s)
\horrule{0.5pt} \\[0.4cm] % Thin top horizontal rule
\huge CS 6220  Data Mining \textemdash~Assignment 2\\ % The assignment title
\Large \textbf{DUE DATE: January 25, 2022} % The assignment title 
\horrule{2pt} \\[0.5cm] % Thick bottom horizontal rule
}

\author{\textbf{YOUR NAME} \\ \textbf{YOUR LDAP}} % Your name
\date{} % Today's date or a custom date

\begin{document}

\AddToShipoutPictureBG*{%
  \AtPageUpperRight{\raisebox{-\height}{\includegraphics[width=3cm]{images/logo}}}}
\maketitle % Print the title

%%%%%%%%%%%%%%%%%%%%
\section{Map Reduce in Spark}
%%%%%%%%%%%%%%%%%%%%

Write a Spark program that implements a simple “People You Might Know” social network friendship recommendation algorithm. The key idea is that if two people have a lot of mutual friends, then the system should recommend that they connect with each other. 

\subsection{Data}

\begin{itemize}
    \item Associated data file is soc-LiveJournal1Adj.txt in q1/data. 
    \item The file contains the adjacency list and has multiple lines in the following format: \\ \verb"<User><TAB><Friends>"
    \item Here, \verb"<User>" is a unique integer ID corresponding to a unique user and \verb"<Friends>" is a comma separated list of unique IDs corresponding to the friends of the user with the unique ID \verb"<User>". Note that the friendships are mutual (i.e., edges are undirected): if A is friend with B then B is also friend with A. The data provided is consistent with that rule as there is an explicit entry for each side of each edge.
\end{itemize}

\subsection{Algorithm}
Let us use a simple algorithm such that, for each user U, the algorithm recommends N = 10 users who are not already friends with U, but have the most number of mutual friends in common with U. 

\subsection{Output}
\begin{itemize}
    \item The output should contain one line per user in the following format: \\ \verb"<User><TAB><Recommendations>"
    \item Here, \verb"<User>" is a unique ID corresponding to a user and \verb"<Recommendations>" is a comma separated list of unique IDs corresponding to the algorithm’s recommendation of people that <User> might know, ordered in decreasing number of mutual friends. 
    \item \textbf{Note}: The exact number of recommendations per user could be less than 10. If a user has less than 10 second-degree friends, output all of them in decreasing order of the number of mutual friends. If a user has no friends, you can provide an empty list of recommendations. If there are recommended users with the same number of mutual friends, then output those user IDs in numerically ascending order.
\end{itemize}

\subsection{Pipeline sketch}
Please provide a description of how you used Spark to solve this problem. Don’t write more than 3 to 4 sentences for this: we only want a very high-level description of your strategy to tackle this problem. 

\subsection{Tips}
 Use Google Colab to use Spark seamlessly, e.g., copy and adapt the setup cells from Colab 0. 
 
\begin{itemize}
    \item Before submitting a complete application to Spark, you may go line by line, checking the outputs of each step. Command .take(X) should be helpful, if you want to check the first X elements in the RDD. 
    \item For sanity check, your top 10 recommendations for user ID 11 should be: \\ 27552, 7785, 27573, 27574, 27589, 27590, 27600, 27617, 27620, 27667. 
    \item The execution may take a while. Our implementations took around 10 minutes. 
\end{itemize}



%%%%%%%%%%%%%%%%%%%%
\section{Submission Instructions}
%%%%%%%%%%%%%%%%%%%%

\begin{itemize}
    \item Commit your code to main in Github, and provide the link. We will snapshot and download it. If you created your code in Colab, you can download your Colab as an iPython Notebook by going to \verb"Download" and selecting \verb"Download .ipynb". 
    \item Your documentation and code's legibility is part of our grading criterion, so please make sure it's readable.
    \item Include your writeup, including a pipeline description with a diagram.
    \item Include in your writeup the recommendations for the users with following user IDs: 924, 8941, 8942, 9019, 9020, 9021, 9022, 9990, 9992, 9993.
\end{itemize}



\end{document}
